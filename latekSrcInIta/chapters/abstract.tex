\chapter{Abstract}

L’ascesa della tecnologia blockchain ha introdotto livelli senza precedenti di trasparenza finanziaria grazie a registri pubblici come Ethereum. Tuttavia, questa trasparenza solleva anche sfide significative in termini di privacy. \textbf{Tornado Cash}, uno \textit{smart contract} decentralizzato e non custodiale su Ethereum, affronta queste problematiche consentendo agli utenti di offuscare la cronologia delle transazioni attraverso avanzate tecniche crittografiche, come le \textit{zero-knowledge proofs}.  

Fin dal suo lancio, Tornado Cash ha alimentato un acceso dibattito all’interno della comunità blockchain e oltre. Se da un lato rappresenta uno strumento legittimo per rafforzare la privacy e la sovranità finanziaria, dall’altro ha attirato l’attenzione dei regolatori a causa del rischio di utilizzo per attività illecite, come il riciclaggio di denaro e l’elusione delle sanzioni economiche. La controversia ha raggiunto il suo apice nel 2022, quando l’Office of Foreign Assets Control (OFAC) del Dipartimento del Tesoro degli Stati Uniti ha imposto sanzioni contro il protocollo\cite{treasury2023}, segnando uno dei primi interventi legali nei confronti di un sistema basato su \textit{smart contract} decentralizzati. Tuttavia, nel novembre 2024, un tribunale statunitense ha annullato le sanzioni, stabilendo che gli \textit{smart contract} immutabili di Tornado Cash non possono essere considerati proprietà e, di conseguenza, non rientrano nelle restrizioni imposte dall'OFAC.

In questo lavoro andremo ad esporre il funzionamento di Tornado Cash, comprensivo di contratti e DAO. Vedremo le strutture matematiche sosttostanti. Andremo ad implementare i contratti su una Testnet e discuteremo delle controversie legate all'utilizzo del protocollo e lo stato attuale. 
