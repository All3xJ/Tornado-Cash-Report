\chapter{Approfondimento: zk-SNARKs}
\label{cap:zk-snarks}

\section{Circuito}

Un circuito nel contesto di zk-SNARK, usato in Tornado Cash, è una sequenza di vertici e archi che forma un anello chiuso. Più precisamente:

\begin{itemize}
    \item Il vertice iniziale e il vertice finale coincidono
    \item Non ci sono ripetizioni di archi o vertici intermedi (eccetto per il vertice iniziale/finale)
    \item La sequenza è chiusa, formando una sorta di anello
\end{itemize}

In pratica, il circuito è un grafo ciclico, dove i dati fluiscono attraverso i vertici (operazioni) collegati da archi (relazioni tra variabili).

In Tornado Cash i circuiti vengono usati come struttura per la generazione di witnesses.

Circom è un linguaggio dichiarativo basato su R1CS (Rank-1 Constraint System)\cite{ben2013snarks}, un modello matematico per rappresentare circuiti aritmetici. Un circuito scritto in Circom è quindi una sequenza di operazioni che vengono tradotte/compilate in un sistema di vincoli matematici che possono essere utilizzati per generare prove zk-SNARKs o zk-STARKs.

Un R1CS è un modo per esprimere un calcolo o un problema aritmetico come un insieme di equazioni quadratiche di rango 1. In pratica, trasforma un programma o una relazione matematica in un sistema di vincoli che può essere verificato in modo efficiente. 

Ogni vincolo è della forma: $A\cdot x\cdot B\cdot x = C\cdot x$ dove:

\begin{itemize}
    \item x è un vettore di variabili (che include sia le variabili di input che quelle intermedie e di output del calcolo).
    \item A, B e C sono vettori di coefficienti (spesso rappresentati come matrici quando si considerano più vincoli).
    Il simbolo "$\cdot$" indica il prodotto scalare.
\end{itemize}

In questo contesto, "rango 1" significa che ogni vincolo può essere espresso come un prodotto di due espressioni lineari ($A\cdot x$ e $B\cdot x$) che, moltiplicate, devono eguagliare una terza espressione lineare ($C\cdot x$).

La compilazione del circuito avviene nella seguente maniera:

\begin{enumerate}
    \item Un calcolo viene scomposto in una sequenza di operazioni di base.
    \item Ogni operazione viene tradotta in un vincolo R1CS.
    \item Il vettore x contiene una "testimonianza" (witness), ossia i valori che soddisfano tutti i vincoli contemporaneamente.
\end{enumerate}

\subsection{Generazione del witness}

\begin{lstlisting}
    // Verifies that commitment that corresponds to given secret and nullifier is included in the merkle tree of deposits
template Withdraw(levels) {
    signal input root;
    signal input nullifierHash;
    signal input recipient; // not taking part in any computations
    signal input relayer;  // not taking part in any computations
    signal input fee;      // not taking part in any computations
    signal input refund;   // not taking part in any computations
    signal private input nullifier;
    signal private input secret;
    signal private input pathElements[levels];
    signal private input pathIndices[levels];

    component hasher = CommitmentHasher();
    hasher.nullifier <== nullifier;
    hasher.secret <== secret;
    hasher.nullifierHash === nullifierHash;

    component tree = MerkleTreeChecker(levels);
    tree.leaf <== hasher.commitment;
    tree.root <== root;
    for (var i = 0; i < levels; i++) {
        tree.pathElements[i] <== pathElements[i];
        tree.pathIndices[i] <== pathIndices[i];
    }

    // Add hidden signals to make sure that tampering with recipient or fee will invalidate the snark proof
    // Most likely it is not required, but it's better to stay on the safe side and it only takes 2 constraints
    // Squares are used to prevent optimizer from removing those constraints
    signal recipientSquare;
    signal feeSquare;
    signal relayerSquare;
    signal refundSquare;
    recipientSquare <== recipient * recipient;
    feeSquare <== fee * fee;
    relayerSquare <== relayer * relayer;
    refundSquare <== refund * refund;
}
\end{lstlisting}

Il circuito Withdraw ha lo scopo di:
\begin{enumerate}
    \item Verificare che un certo commitment (basato su nullifier e secret) sia valido e coerente con il nullifierHash pubblico.
    \item Confermare che questo commitment sia presente in un albero di Merkle con una certa radice (root).
    \item Assicurare che parametri aggiuntivi (come recipient, fee, ecc.) siano vincolati nella prova.
\end{enumerate}

e lo fa nella seguente maniera:

\begin{enumerate}
    \item \textbf{Nullifier Hash Check}: il circuito prende gli input privati nullifier e secret e li passa a un componente (CommitmentHasher) che calcola due hash Pedersen:
        \begin{itemize}
            \item Il \textbf{commitment hash}: l'hash dell'intero messaggio di commitment (che combina nullifier e secret).
            \item Il \textbf{nullifier hash}: l'hash del solo nullifier.
        \end{itemize}
        \textbf{Verifica}: Il circuito confronta il nullifier hash calcolato e il nullifierHash fornito come input pubblico, e impone che siano uguali (hasher.nullifierHash === nullifierHash)
    \item \textbf{Merkle Tree Check}: il circuito usa il commitment hash (calcolato sopra), la radice pubblica dell'albero di Merkle (root), e gli input privati pathElements e pathIndices (il percorso nell'albero) per verificare che il commitment sia una foglia valida dell'albero.
    \begin{itemize}
        \item Parte dalla foglia (il commitment hash) e usa un componente chiamato Muxer per combinare il commitment con il primo elemento del percorso (pathElements[0]).
        \item L'input pathIndices[0] (0 o 1) indica se il commitment hash è a sinistra o a destra nel calcolo dell'hash del livello successivo, usando un hasher MiMC.
        \item Questo processo si ripete livello per livello, usando l'hash del livello precedente e il successivo pathElements[i], fino a calcolare un hash finale
        \item Il circuito verifica che questo hash finale sia uguale al Merkle root pubblico (tree.root $<==$ root).
    \end{itemize}
    \item \textbf{Extra Withdrawal Parameter Check}: il circuito prende gli input pubblici recipient, relayer, fee, e refund e calcola i loro quadrati (es. recipientSquare $<==$ recipient * recipient).  Anche se questi valori non sono usati nei calcoli principali, aggiungerli al witness crea vincoli che "bloccano" i parametri della transazione. Se qualcuno cercasse di modificarli dopo la generazione della prova, i vincoli non sarebbero più soddisfatti e la prova risulterebbe invalida.
\end{enumerate}

Dopo aver creato il circuito, il compito di generare il witness viene affidato alla funzione calculateWitness (\url{https://github.com/tornadocash/snarkjs/blob/master/src/calculateWitness.js}). Questa funzione prende in input il circuito e l'insieme di segnali di input forniti dall’utente. Il suo obiettivo è calcolare e restituire un array, che contiene i valori di tutti i segnali del circuito necessari per soddisfare i vincoli definiti.

Generare il witness significa eseguire il circuito passo per passo: assegnare gli input, calcolare i valori intermedi, e verificare che tutti i vincoli siano rispettati. Il risultato è una lista ordinata di numeri che rappresenta la "soluzione" del circuito, pronta per essere usata come base per una ZKP.

Questo witness viene poi passato a un sistema come SNARK (snarkjs) per generare una prova crittografica compatta. Tale prova può essere verificata pubblicamente da chiunque per confermare che i vincoli del circuito sono soddisfatti, senza rivelare i dati privati contenuti nel witness.

\subsection{Calcolo della proof}

Groth16 è un sistema di zero-knowledge proofs di tipo zk-SNARK. È stato sviluppato da Jens Groth nel 2016 ed è uno degli schemi più utilizzati per dimostrare che un calcolo è stato eseguito correttamente senza rivelare i dati privati usati nel calcolo.

Groth16 trasforma un circuito aritmetico in un Quadratic Arithmetic Program (QAP), che rappresenta i vincoli come polinomi. La prova dimostra che esiste un witness 
$w$ che soddisfa l'equazione polinomiale del QAP, senza rivelarlo. 

La prova è composta da tre elementi $\pi_A\in G_1,\pi_B\in G_2$ e $\pi_C\in G_1$ dove $G_1$ e $G_2$ sono gruppi su una curva ellittca (quella utilizzata da Tornado Cash è BN254).

\subsubsection{Setup e QAP}

Il circuito viene convertito in un QAP con polinomi \(A_i(x)\), \(B_i(x)\), \(C_i(x)\) per ogni segnale \(w_i\) nel witness, e un polinomio divisore \(Z(x)\), detto \emph{target polynomial}. La condizione del QAP è:
\begin{equation}
    \sum_{i=0}^{m-1} w_i A_i(x) \cdot \sum_{i=0}^{m-1} w_i B_i(x) = \sum_{i=0}^{m-1} w_i C_i(x) + H(x) \cdot Z(x)
\end{equation}
dove:
\begin{itemize}
    \item \(m\): numero totale di segnali,
    \item \(w_i\): valori del witness,
    \item \(H(x)\): polinomio quoziente, calcolato dal prover.
\end{itemize}

\subsubsection{Proving Key}
La proving key contiene valori precalcolati:
\begin{itemize}
    \item \(\alpha, \beta, \delta \in \mathbb{F}_r\): scalari casuali,
    \item \([A_i]_1 = [A_i(\tau)]_1\), \([B_i]_1 = [B_i(\tau)]_1\), \([B_i]_2 = [B_i(\tau)]_2\), \([C_i]_1 = [C_i(\tau)]_1\): valutazioni dei polinomi in un punto segreto \(\tau\), mappate su \(G_1\) e \(G_2\),
    \item \([\alpha]_1\), \([\beta]_1\), \([\beta]_2\), \([\delta]_1\), \([\delta]_2\): punti generati da \(\alpha\), \(\beta\), \(\delta\),
    \item \([H_i]_1\): punti per i coefficienti di \(H(x)\), dove \(H(x) = \sum h_i x^i\).
\end{itemize}

\subsection{Generazione della prova}
La prova combina il witness \(w\) con la proving key, aggiungendo randomizzazione per garantire zero-knowledge.

\subsection*{1. Calcolo iniziale dei termini base}
\begin{itemize}
    \item \(A\):
    \begin{equation}
        A = \sum_{i=0}^{m-1} w_i [A_i(\tau)]_1
    \end{equation}
    Somma dei segnali \(w_i\) moltiplicati per i punti \([A_i]_1\).
    
    \item \(B\):
    \begin{equation}
        B_1 = \sum_{i=0}^{m-1} w_i [B_i(\tau)]_1, \quad B_2 = \sum_{i=0}^{m-1} w_i [B_i(\tau)]_2
    \end{equation}
    Due somme: una in \(G_1\) (\(B_1\)) e una in \(G_2\) (\(B_2\)).
    
    \item \(C\):
    \begin{equation}
        C = \sum_{i=n_{pub}+1}^{m-1} w_i [C_i(\tau)]_1
    \end{equation}
    Somma dei segnali privati (escludendo i pubblici e il segnale "one").
    
    \item \(H\):
    \begin{equation}
        H = \sum_{i=0}^{d-1} h_i [x^i]_1, \quad \text{dove} \quad H(x) = \frac{A(x) \cdot B(x) - C(x)}{Z(x)}
    \end{equation}
    \(H(x)\) è il polinomio quoziente, con \(d\) grado di \(Z(x)\).
\end{itemize}

Si scelgono due scalari casuali \(r, s \in \mathbb{F}_r\), usati per mascherare la prova.

\begin{itemize}
    \item \(\pi_A\):
    \begin{equation}
        \pi_A = A + [\alpha]_1 + r \cdot [\delta]_1
    \end{equation}
    
    \item \(\pi_B\):
    \begin{equation}
        \pi_B = B_2 + [\beta]_2 + s \cdot [\delta]_2
    \end{equation}
    
    \item \(\pi_C\):
    \begin{equation}
        \pi_C = C + H + s \cdot \pi_A + r \cdot B_1 - r \cdot s \cdot [\delta]_1
    \end{equation}
\end{itemize}

\subsection{Verifica}
La verifica della prova in Groth16 controlla se una prova \((\pi_A, \pi_B, \pi_C)\), generata per un \emph{Quadratic Arithmetic Program} (QAP), soddisfa i vincoli di un circuito aritmetico senza rivelare il witness \(w\). La verifica si basa su un'equazione di accoppiamento (\emph{pairing}) sulle curve ellittiche \(G_1\) e \(G_2\), utilizzando la \emph{verifying key} e i segnali pubblici del witness.

\subsubsection{Equazione di verifica:}
La verifica controlla l'equazione di accoppiamento:
\begin{equation}
    e(\pi_A, \pi_B) = e([\alpha]_1, [\beta]_2) \cdot e\left(\sum_{i=0}^{n_{pub}} w_i [IC_i]_1, [\gamma]_2\right) \cdot e(\pi_C, [\delta]_2)
\end{equation}
dove \(e: G_1 \times G_2 \to G_T\) è una funzione di accoppiamento bilineare. 

La prova è valida se l'equazione è soddisfatta, cioè se entrambi i lati sono uguali in \(G_T\).

\subsubsection{Calcolo della combinazione lineare \(vk_x\):}
\begin{equation}
    vk_x = [IC_0]_1 + \sum_{i=1}^{n_{pub}} w_i [IC_i]_1
\end{equation}
\(vk_x\) rappresenta i segnali pubblici pesati dai punti della verifying key.

\subsubsection*{Verifica dell'accoppiamento}
Si calcola:
\begin{itemize}
    \item Lato sinistro: \(e(-\pi_A, \pi_B)\),
    \item Lato destro: \(e([\alpha]_1, [\beta]_2) \cdot e(vk_x, [\gamma]_2) \cdot e(\pi_C, [\delta]_2)\),
\end{itemize}
dove \(-\pi_A\) è la negazione di \(\pi_A\) in \(G_1\). La prova è valida se:
\begin{equation}
    e(-\pi_A, \pi_B) = e([\alpha]_1, [\beta]_2) \cdot e(vk_x, [\gamma]_2) \cdot e(\pi_C, [\delta]_2)
\end{equation}

